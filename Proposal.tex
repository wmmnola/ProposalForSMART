\documentclass{article}
\usepackage[utf8]{inputenc}
\usepackage{lmodern}
\usepackage{amsmath}
\usepackage{hyperref}
\title{Proposal for A SMART Scheduling System}
\author{Wade Marshall }
\date{May 2017}

\begin{document}

\maketitle

\section{Introduction}
Spreadsheets are an archaic and slow organization method. While useful for small amounts of data, once we introduce large amounts of data, it
becomes a mess. Such is the situation we are presently faced with the SMART club scheduling system. The spreadsheets which manage it, are no
longer convenient. We run a spreadsheet script, which no one understands anymore, and it constantly breaks down. The hope of this proposal is to 
streamline the process, to dump spreadsheets once and for all. 

The answer is simple. Create a Database. A database would allow for easy manipulation of large amounts of data, and can be coupled with a easy
to use user-interface. This strategy also scales well. Databases have no "limit," so long as we gear the software so there is no hard limit. This 
will be incredibly important in the next few years, as the school attempts to expand it's infrastructure to accommodate more students. The 
current technology \emph{does not scale.}

The detraction is the start up cost. To migrate the current system will take considerable work; however, it is not enough to warrant continuing
to run off of spreadsheets. 

\section{Back-End Development}
\subsection{General Overview}
We will be developing the database using SQL. Most of the requests will be done using PHP, there is a possibility to use node.js as an alternative back end language, but PHP has specific advantages in that it can be embedded in web pages, versus running on a server. 

In conjunction with PHP, mySQL will allow dynamic storage of the students and their clubs. We can make specific requests for students, which 
will
allow easier schedules to be made for teachers, and for people. Simple SQL requests will be made, in order to simplify the processing of the 
queries. 

PHP has been chosen because of it's nativity regarding SQL. Not only that, but it will integrate seamlessly with the javascript front end. In 
order to produce a totally seamless back end. We need a simple algorithm to assign people to SMART clubs. Therefore we need an assignment algorithm. 

Ideally the Stable Marriage Problem Algorithm(SMA)\footnote{The Stable Marriage Problem is a well defined problem in Mathematics, Computer 
Science 
and economics. Ultimately the goal is to assign one element of a set to an element of another set. The solution is ingenious, and the algorithm 
has a completion time of $O(n^2)$. Please see Wikipedia for more information.} would be the optimal solution, as its efficent and elegant.
However, we are aiming to minimize the role of teachers in the survey, and using the SMA would require the teachers to rank every student. 
That might be a little too much to ask of teachers. 

Therefore we must minimize teacher interaction, one way of doing this is to eliminate it completely. 

\subsection{Survey Sorting and Class Assignment}
Not only will we need to read student's clubs, and grade level, but we will need to change them manually. This can be accomplished with the 
insert command, and should be rather trivial to implement. The harder task will be the assignment of students at the start of the year based on 
survey data. \emph{This is the main challenge of the whole project.}

We will need a mechanism to input survey data\footnote{For more information on survey data, please refer to section 2}. This data should be 
organized in a preference list. 
\section{Front End}
For the trial period, I plan to create a simple HTML and Javascript web page in order to suffice as a front-end. In order to minimize 
confusion, I plan on only having one web page, which will be separated into sections based on queries and changes. 
\section{Plans for the Future}

\end{document}