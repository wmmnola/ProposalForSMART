\documentclass{article}
\usepackage[utf8]{inputenc}
\usepackage{lmodern}
\title{Proposal for A SMART Scheduling System}
\author{Wade Marshall }
\date{May 2017}

\begin{document}

\maketitle

\section{Introduction}
Spreadsheets are an archaic and slow organization method. While useful for small amounts of data, once we introduce large amounts of data, it
becomes a mess. Such is the situation we are presently faced with the SMART club scheduling system. The spreadsheets which manage it, are no
longer convenient. We run a spreadsheet script, which no one understands anymore, and it constantly breaks down. The hope of this proposal is to 
streamline the process, to dump spreadsheets once and for all. 

The answer is simple. Create a Database. A database would allow for easy manipulation of large amounts of data, and can be coupled with a easy
to use user-interface. This strategy also scales well. Databases have no "limit," so long as we gear the software so there is no hard limit. This 
will be incredibly important in the next few years, as the school attempts to expand it's infrastructure to accommodate more students. The 
current technology \emph{does not scale.}

The detraction is the start up cost. To migrate the current system will take considerable work; however, it is not enough to warrant continuing
to run off of spreadsheets. 

\section{Back-End Development}
We will be developing the database using SQL. Most of the requests will be done using PHP, there is a possibility to use node.js as an alternative back end language, but PHP has specific advantages in that it can be embedded in web pages, versus running on a server. 

In conjunction with PHP, mySQL will allow dynamic storage of the students and their clubs. We can make specific requests for students, which will
allow easier schedules to be made for teachers, and for people 
\section{Front End}

\section{Plans for the Future}

\end{document}