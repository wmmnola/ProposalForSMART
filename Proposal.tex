\documentclass{article}
\usepackage[utf8]{inputenc}
\usepackage{lmodern}
\usepackage{amsmath}
\usepackage{amsfonts}
\usepackage{amssymb}
\usepackage{hyperref}
\title{Proposal for A SMART Scheduling System}
\author{Wade Marshall }
\date{May 2017}

\begin{document}

\maketitle

\section{Introduction}
Spreadsheets are an archaic and slow organization method. While useful for small amounts of data, once we introduce large amounts of data, it
becomes a mess. Such is the situation we are presently faced with the SMART club scheduling system. The spreadsheets which manage it, are no
longer convenient. We run a spreadsheet script, which no one understands anymore, and it constantly breaks down. The hope of this proposal is to 
streamline the process, to dump spreadsheets once and for all. 

The answer is simple. Create a Database. A database would allow for easy manipulation of large amounts of data, and can be coupled with a easy
to use user-interface. This strategy also scales well. Databases have no "limit," so long as we gear the software so there is no hard limit. This 
will be incredibly important in the next few years, as the school attempts to expand it's infrastructure to accommodate more students. The 
current technology \emph{does not scale.}

The detraction is the start up cost. To migrate the current system will take considerable work; however, it is not enough to warrant continuing
to run off of spreadsheets. 

\section{Back-End Development}
\subsection{General Overview}
We will be developing the database using SQL. Most of the requests will be done using PHP, there is a possibility to use node.js as an alternative back end language, but PHP has specific advantages in that it can be embedded in web pages, versus running on a server. 

In conjunction with PHP, mySQL will allow dynamic storage of the students and their clubs. We can make specific requests for students, which 
will
allow easier schedules to be made for teachers, and for people. Simple SQL requests will be made, in order to simplify the processing of the 
queries. 

PHP has been chosen because of it's nativity regarding SQL. Not only that, but it will integrate seamlessly with the javascript front end. In 
order to produce a totally seamless back end. We need a simple algorithm to assign people to SMART clubs. Therefore we need an assignment algorithm. 


\subsection{Survey Sorting and Class Assignment}
Not only will we need to read student's clubs, and grade level, but we will need to change them manually. This can be accomplished with the 
insert command, and should be rather trivial to implement. The harder task will be the assignment of students at the start of the year based on 
survey data. \emph{This is the main challenge of the whole project.}

We will need a mechanism to input survey data\footnote{For more information on survey data, please refer to section 3: The Front End}. This data should be 
organized in a preference list. Then we can apply a sorting algorithm to the data to sort the students. 

Ideally the Stable Marriage Problem Algorithm(SMA)\footnote{The Stable Marriage Problem is a well defined problem in Mathematics, Computer 
Science 
and economics. Ultimately the goal is to assign one element of a set to an element of another set. The solution is ingenious, and the algorithm 
has a completion time of $O(n^2)$. Please see Wikipedia for more information.} would be the optimal solution, as its efficient and elegant.
However, we are aiming to minimize the role of teachers in the survey, and using the SMA would require the teachers to rank every student. 
That might be a little too much to ask of teachers. 

Therefore we must minimize teacher interaction, one way of doing this is to eliminate it completely. Therefore we must devise an algorithm that only looks at student preferences. 

I have devised a simple algorithm to do just this. We will pick a random student from the list of students, then we will assign them to a SMART
Club according to their listed preference, but to ensure that the class is not full. Then we will remove the student from the list of students to assign and then repeat\footnote{See Appendix subsection 1 for a mathematically rigorous definition of the algorithm}. The advantage of this algorithim is that it is just as efficient as the SMA\footnote{See Appendix subsection 2 for the classification of the algorithm}.
\section{Front End}
For the trial period, I plan to create a simple HTML and Javascript web page in order to suffice as a front-end. In order to minimize 
confusion, I plan on only having one web page, which will be separated into sections based on queries and changes. Principally we need the 
front end to do three things: list students and their smarts, change student's smarts, and bulk assign smarts. A simple system of queries and inserts should be able to accomplish the first two functions; however, the third function will need to be a combination of user input and back end. We will need to administer a survey at the start of the year, which will require students to rank each SMART club, then we will feed the data into the website, which will then run the algorithm described in the previous section. 

\section{Appendix}
\subsection{Mathematical Definition of Our Sorting Algorithm. }

Let $\varsigma = $ the set of students

Let $\vartheta_n = $ the set of smart clubs ranked in order for the nth student.

Let $ C_{capacity} $ be the capacity of a given smart club. 

Let $ C_{members} $ be the number of members in a given club $C$.

I will use the notation $\rightarrow$ to denote a student being assigned into a class. This will increase the $C_{members}$ of a given class 
by 1.

I will also being appropriating computer syntax for sets. So $A[n]$ is the n-th element of set A.  

\begin{equation}x \in \varsigma_0 \end{equation}
\begin{equation}
\forall \vartheta_x: C_{capacity} > C_{members} 
\end{equation}
\begin{equation}
x \rightarrow \vartheta_x[1]
\end{equation}
\begin{equation}
\varsigma_2 = \varsigma_1
\end{equation}
\begin{equation}
x \notin \varsigma_2
\end{equation}

\subsection{Run time of the Algorithm.}
An important part of evaluating any algorithm is to put it into Big-O Notation. This gives us a rough estimate for the run time and class of 
the algorithm. Let $n = $the number of students. Let $m = $the number of smart clubs. The run time of the algorithm is given by the following equation 
\begin{equation}
T(n) = k_1n^2 + k_2mn
\end{equation}
Where $k_1$ and $k_2$ are arbitrary constants related to the completion time of the algorithm. We can now deduce, based on this equation, the 
class of the algorithm. 

As $n \to \infty$ and $m \to \infty$ the $n^2$ will grow at a rate of $2k_1n$, while $nm$ will grow at a constant rate of $k_2$. Therefore the
$n^2$ term will grow to dominate $T(n)$
\begin{equation}
T(n) \in O(n^2)
\end{equation}
\end{document}